\chapter{Output}
\todo{think of a better title}

\section{Introduction}
Once you have your data set up the way you would like you will want 
to export it in some sort of form that allows you to use it.

\section{Working with Geometries}\label{geoms}
\todo{decide where this should go}
Within \GeoTools geometries are represented by \ac{JTS} Geometries that contain the actual positions as Coordinates.
Sometimes when exporting geometries you would like to produce a plain text string that represents the geometry, if for no other reason than it's easier to see what is happening than starting up a debugger. 
This sort of interoperability issue comes up often enough that there is a standard way of doing it (well two in fact). The most common format is \ac{WKT}.

\lstinputlisting[firstline=38,firstnumber=38,lastline=40,label=wkt,caption={Exporting WKT from a geometry}]{../modules/output/src/main/java/org/ianturton/cookbook/output/ExportGeometries.java}

\Cref{wkt} shows just how simple this process is. This will produce output like \cref{wktoutput}. WKT is kept deliberately simple so that writing a program that can read it in or write it out is easy. The standard was originally written by the \ac{OGC} as part of Simple Feature specification, it is now defined in the ISO 13249 ``Information technology -- Database languages -- SQL multimedia and application packages -- Part 3: Spatial'' standard.

\begin{figure}[h]
\begin{spverbatim}
POINT (10.1 22.2) 
LINESTRING (1 1, 2 3, 4.4 5.2)
POLYGON ((1 1, 2 3, 4.4 5.2, 1 1))
MULTIPOLYGON (((1 1, 2 3, 4.4 5.2, 1 1)), 
    ((10 10, 20 30, 40.4 50.2, 10 10)))
\end{spverbatim}
\caption{Some examples of WKT}\label{wktoutput}
\end{figure}

One important point to notice is that \spverb!POLYGON! has two parenthesis at the start and end, this allows you to specify holes in your polygon (think islands in a lake) which are represented like \spverb!POLYGON ((35 10, 45 45, 15 40, 10 20, 35 10),(20 30, 35 35, 30 20, 20 30))!. 

\section{Creating an Image of the Map}
Assuming that you have created a map that you like with the correct styling and area of interest 
set then you might want to produce a picture of it. \Cref{save} shows how to do this.

\lstinputlisting[firstline=27,firstnumber=27,label=save,caption=Save a picture of the map.]{../modules/output/src/main/java/org/ianturton/cookbook/output/SaveMapAsImage.java}

\section{Create a New Shapefile}

\lstinputlisting[firstline=40,firstnumber=40,label=shapefile,caption=Save a shapefile.]{../modules/output/src/main/java/org/ianturton/cookbook/output/WriteShapefile.java}

